\section{Baza XML}



\subsection{Schemat dokumentu}



\begin{figure}[h]
 \centering
 \subfigure[element text]{
 \includegraphics[width=120px]{../diagramy/schema_text.pdf}
 }
   \subfigure[element tablet]{
  \includegraphics[width=120px]{../diagramy/schema_tablet.pdf}
 }
 \subfigure[element tablets]{
  \includegraphics[width=120px]{../diagramy/schema_tablets.pdf}
 }
 \caption{Schematy poszczególnych elementów (kolor pomarańczowy oznacza atrybuty, a niebieski elementy)}
\end{figure}



Schemat dokumentu zamieszczony powyżej jest graficznym przedstawieniem dokumentu XML Schema (dodatek A) wygenerowanym przez program Oxygen.\\
Schemat XML jest częściowo przepisaniem diagramu encji bazy PostgreSQL na język XML Schema. Przy tworzeniu schematu, podobnie jak w bazie relacyjnej, skorzystałyśmy z pomysłu dra Wojciecha Jaworskiego, aby przedstawić treść tabliczki w formie grafu. Każda krawędź tego grafu (odpowiadająca odczytowi) jest oddzielnym elementem, zawierającym atrybuty node1 oraz node2 - numery węzłów grafu, pomiędzy którymi się znajduje oraz atrybut symbol oznaczający typ symbolu znajdującego się na danej krawędzi. Dodatkowo przechowujemy treść tabliczki w formie napisu (element show). Metadane tabliczek przechowywane są w postaci podelementów elementu tablet.


\subsection{Translator\_xml}
Do przeszukiwania dokumentu XML wykorzystujemy język XQuery, który jest częścią rekomendacji W3C dotyczącej XML.\\
Proste zapytanie TQL tłumaczymy na pojedynczą konstrukcję FLWOR (For Let Where Order by Return).\\

\subsubsection{Stałe fragmenty}
Każde zapytanie w części For zawiera:
	\begin{verbatim}
	FOR $tablet IN .//tablet
\end{verbatim}
a w części Return:
  \begin{verbatim}RETURN <tablet>
		{$tablet/idCDLI}
		{$tablet/publication}
		{$tablet/provenience}
		{$tablet/period}
		{$tablet/measurements}
		{$tablet/genre}
		{$tablet/subgenre}
		{$tablet/collection}
		{$tablet/museum}
		{$tablet/text/show}
		<seq>...</seq>
	</tablet>
\end{verbatim}
Zawartość elementu seq zależy od ilości sekwencji, po których wyszukujemy. 

\subsubsection{Zapytania proste}

\begin{longtable}{|p{2.5in}|p{3.5in}|}
\hline
{\bf Konstrukcja} & {\bf Tłumaczenie na XQuery}\\
\hline
\endhead
provenience: wartosc & \begin{verbatim}fn:matches($tablet/provenience,'^wartosc$')\end{verbatim}
\\
\hline
publication: wartosc & \begin{verbatim}fn:matches($tablet/publication,'^wartosc$')\end{verbatim}
\\
\hline
period: wartosc & \begin{verbatim}fn:matches($tablet/period,'^wartosc$')\end{verbatim}
\\
\hline
genre: wartosc & \begin{verbatim}(fn:matches($tablet/genre,'^wartosc$')
or fn:matches($tablet/subgenre,'^wartosc$'))\end{verbatim}
\\
\hline
cdli\_id: wartosc & \begin{verbatim}fn:matches($tablet/idCDLI,'^wartosc$')\end{verbatim}
\\
\hline
\end{longtable}


\subsubsection{Treść tabliczki}
Każda sekwencja, po której wyszukujemy powoduje dodanie do zapytania następujących konstrukcji:
\begin{itemize}
\item{do części Let:}
\begin{verbatim}
let $seq <id_sekw> := (
	for $edge_end in $tablet//edge
	for $edge_start in $tablet//edge
	where (
		fn:matches($edge_start,'^<sekw[0]>$')
		and (
			some $edge1 in $tablet//edge[@node1=$edge_start/@node2]
satisfies (fn:matches($edge1,'^<sekw[1]>$')
and ... 
and fn:matches($edge_end,'^<sekw[dl_sekw-1]>$')))))
return <seq<id_sekw>> {$edge_start/@node1} {$edge_end/@node2} </seq<id_sekw>>
\end{verbatim}
\item{do części Where}
\begin{verbatim}
$seq<id_sekw>
\end{verbatim}
\item{do części Return w elemecie seq}
\begin{verbatim}
$seq<id_sekw>
\end{verbatim}
\end{itemize}




\subsubsection{Operatory}
\begin{longtable}{|p{1in}|p{3in}|}
\hline
{\bf Operator} & {\bf Tłumaczenie}\\
\hline
\endhead
/ & \begin{verbatim}(<zapytanie1> or <zapytanie2>) \end{verbatim} \\
\hline
-- & \begin{verbatim}not (<zapytanie_negowane>) \end{verbatim}\\  
\hline
+ & \begin{verbatim}(<zapytanie1> and <zapytanie2>) \end{verbatim}\\ 
\hline
* & \begin{verbatim} .*\end{verbatim}  \\ 
\hline
\end{longtable}

\subsubsection{Zapytania złożone}
Zapytanie złożone, składające się z wielu zapytań prostych tłumaczymy na sekwencję zapytań XQuery połączonych znakiem ','.

\subsection{Database\_xml}
Jako bazę danych wykorzystujemy plik XML, określony w pliku konfiguracyjnym xml.conf. Do wyszukiwania wykorzystujemy procesor XQuery Zorba. Posiada on API m.in. do C++, które pozwala na przekazanie zapytania do bazy oraz przetworzenie wyniku.
