\section{Baza XML}

\subsection{Schemat dokumentu}

\begin{figure}[h]
 \centering
 \subfigure[element text]{
 \includegraphics[width=120px]{../diagramy/schema_text.pdf}
 }
   \subfigure[element tablet]{
  \includegraphics[width=120px]{../diagramy/schema_tablet.pdf}
 }
 \subfigure[element tablets]{
  \includegraphics[width=120px]{../diagramy/schema_tablets.pdf}
 }
 \caption{Schematy poszczególnych elementów (kolor pomarańczowy oznacza atrybuty, a niebieski elementy)}
\end{figure}



Schemat dokumentu zamieszczony powyżej jest graficznym przedstawieniem dokumentu XML Schema (dodatek \ref{appendix:xmlsch}) 
wygenerowanym przez program Oxygen.

Jest on oparty, podobnie jak w bazie postgres, na pomyśle dra Wojciecha Jaworskiego, 
 aby przedstawić treść tabliczki w formie grafu. 
% Przy jego tworzeniu, podobnie jak w bazie relacyjnej, skorzystałyśmy z pomysłu dra Wojciecha Jaworskiego, 
% aby przedstawić treść tabliczki w formie grafu. 
Każda krawędź tego grafu (odpowiadająca odczytowi) jest oddzielnym elementem, zawierającym atrybuty \textit{node1}, \textit{node2} 
i \textit{symbol}.
Atrybuty \textit{node1} oraz \textit{node2} oznaczają numery węzłów grafu,
natomiast atrybut \textit{symbol} to typ symbolu znajdującego się na danej krawędzi. 
Dodatkowo przechowujemy treść tabliczki w formie napisu (element \textit{show}). 

Metadane tabliczek przechowywane są w postaci podelementów elementu \textit{tablet}.


\subsection{Translator\_xml}
% TODO: jest
Do przeszukiwania dokumentu XML wykorzystywany jest język XQuery, który jest częścią rekomendacji W3C dotyczącej XML.\\
Proste zapytanie TQL jest tłumaczone na pojedynczą konstrukcję FLWOR (For Let Where Order by Return).\\

\subsubsection{Stałe fragmenty zapytania}
Każde zapytanie w części For zawiera:
	\begin{verbatim}
	FOR $tablet IN .//tablet
\end{verbatim}
a w części Return:
  \begin{verbatim}RETURN <tablet>
		{$tablet/idCDLI}
		{$tablet/publication}
		{$tablet/provenience}
		{$tablet/period}
		{$tablet/measurements}
		{$tablet/genre}
		{$tablet/subgenre}
		{$tablet/collection}
		{$tablet/museum}
		{$tablet/text/show}
		<seq>...</seq>
	</tablet>
\end{verbatim}
Zawartość elementu seq zależy od ilości sekwencji, po których wyszukujemy. 

\subsubsection{Tłumaczenie zapytań o atrybuty tabliczki}

\begin{longtable}{|p{2.5in}|p{3.5in}|}
\hline
{\bf Konstrukcja} & {\bf Tłumaczenie na XQuery}\\
\hline
\endhead
provenience: wartosc & \begin{verbatim}fn:matches($tablet/provenience,'^wartosc$')\end{verbatim}
\\
\hline
publication: wartosc & \begin{verbatim}fn:matches($tablet/publication,'^wartosc$')\end{verbatim}
\\
\hline
period: wartosc & \begin{verbatim}fn:matches($tablet/period,'^wartosc$')\end{verbatim}
\\
\hline
genre: wartosc & \begin{verbatim}(fn:matches($tablet/genre,'^wartosc$')
or fn:matches($tablet/subgenre,'^wartosc$'))\end{verbatim}
\\
\hline
cdli\_id: wartosc & \begin{verbatim}fn:matches($tablet/idCDLI,'^wartosc$')\end{verbatim}
\\
\hline
\end{longtable}


\subsubsection{Tłumaczenie zapytań o treść tabliczki}
Każda sekwencja, po której wyszukujemy powoduje dodanie do zapytania następujących konstrukcji:
\begin{itemize}
\item{do części Let:}
\begin{verbatim}
let $seq <id_sekw> := (
	for $edge_end in $tablet//edge
	for $edge_start in $tablet//edge
	where (
		fn:matches($edge_start,'^<sekw[0]>$')
		and (
			some $edge1 in $tablet//edge[@node1=$edge_start/@node2]
satisfies (fn:matches($edge1,'^<sekw[1]>$')
and ... 
and fn:matches($edge_end,'^<sekw[dl_sekw-1]>$')))))
return <seq<id_sekw>> {$edge_start/@node1} {$edge_end/@node2} </seq<id_sekw>>
\end{verbatim}
\item{do części Where}
\begin{verbatim}
$seq<id_sekw>
\end{verbatim}
\item{do części Return w elemecie seq}
\begin{verbatim}
$seq<id_sekw>
\end{verbatim}
\end{itemize}




\subsubsection{Tłumaczenie operatorów}
Poniższe tłumaczenia dotyczą zarówno konstrukcji prostych jak i złożonych.

\begin{longtable}{|p{1in}|p{3in}|}
\hline
{\bf Operator} & {\bf Tłumaczenie}\\
\hline
\endhead
/ & \begin{verbatim}(<zapytanie1> or <zapytanie2>) \end{verbatim} \\
\hline
-- & \begin{verbatim}not (<zapytanie_negowane>) \end{verbatim}\\  
\hline
+ & \begin{verbatim}(<zapytanie1> and <zapytanie2>) \end{verbatim}\\ 
\hline
* & \begin{verbatim} .*\end{verbatim}  \\ 
\hline
\end{longtable}

\subsubsection{Zapytania złożone}
Zapytanie złożone, składające się z wielu zapytań prostych tłumaczymy na sekwencję zapytań XQuery połączonych znakiem ','.

\subsection{Database\_xml}
Odpowiada za wywołanie zapytania i zapisanie wyniku do struktury Tablets.
Jako bazę danych wykorzystujemy plik XML, określony w pliku konfiguracyjnym xml.conf. 
Do wyszukiwania wykorzystujemy procesor XQuery Zorba \cite{zorba}. 
Posiada on API m.in. do C++, które pozwala na przekazanie zapytania do bazy oraz przetworzenie wyniku.
