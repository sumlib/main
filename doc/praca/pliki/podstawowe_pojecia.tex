\chapter{Podstawowe pojęcia}\label{r:pojecia}
\section{Pojęcia dziedzinowe}
\begin{description}
 \item[Sumerolog] - naukowiec zajmujący się odczytywaniem pisma klinowego w języku sumeryjskim. Na potrzeby tej pracy
		      to pojęcie jest rozszerzone do wszystkich ludzi zajmujących się odczytywaniem tabliczek klinowych (także w innych językach) i czerpiących z nich wiedzę historyczną.
 \item[Tabliczka] - w tej pracy tabliczka będzie oznaczała tabliczkę klinową w wersji elektronicznej 
		  (chyba, że zostanie zaznaczone inaczej). Dla rozróżnienia, kiedy będziemy mówić o ``prawdziwej'', 
		  glinianej tabliczce, będziemy używać pojęcia \textbf{gliniana tabliczka}
 \item[Prowiniencja] - pojęcie używane przez sumerologów, oznacza miejsce pochodzenia/znalezienia glinianej tabliczki
 \item[Kliny] - znaki występujące na glinianych tabliczkach.
 \item[Odczyty] - sposób transkrypcji klinów. Tabliczki elektroniczne są zapisane za pomocą odczytów. Jeden klin może zostać odczytany na wiele różnych sposobów, podobnie jeden odczyt może być zapisany różnymi sekwencjami klinów.
 \item[Pieczęć] - część tabliczki zawierająca znak rozpoznawczy autora
\end{description}

\section{Pojęcia informatyczne}
\begin{description}
 \item[Język] - nie mamy pojęcia jak to zdefiniować
 \item[Gramatyka] - formalny opis struktury składniowej i leksykalnej języka
%TODO: nieterminale??
 \item[Struktura leksykalna języka] - opis słów kluczowych, znaków specjalnych i symboli pomocniczych (nieterminali).
 \item[Struktura składniowa języka] - opis składni języka, dozwolonych konstrukcji.
 \item[Semantyka] - znaczenie konkretnych konstrukcji języka.
 \end{description}
\section{Pojęcia paradygmatyczne}
\begin{description}
 \item[Język dziedzinowy (Domain Specific Language, DSL)] - język programowania dostosowany do dziedziny problemu, którym się zajmuje. 
 \end{description}