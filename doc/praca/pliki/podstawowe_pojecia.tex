\chapter{Podstawowe pojęcia}\label{r:pojecia}
\section{Pojęcia dziedzinowe}
\begin{description}
 \item[Sumerolodzy] - ludzie, którzy zajmują się odczytywaniem pisma klinowego w języku sumeryjskim. Na potrzeby tej pracy
		      to pojęcie jest rozszerzone do wszystkich ludzi zajmujących się odczytywaniem tabliczek sumeryjskich 
		      i wyciąganiem z nich wiedzy historycznej.
 \item[Tabliczka] - w tej pracy tabliczka będzie oznaczała tabliczkę sumeryjską w wersji elektronicznej 
		  (chyba, że zostanie zaznaczone inaczej). Dla rozróżnienia, kiedy będziemy mówić o ``prawdziwej'', 
		  glinianej tabliczce, będziemy używać pojęcia \textbf{gliniana tabliczka}
 \item[Prowiniencja] - pojęcie używane przez sumerologów, oznacza miejsce pochodzenia/znalezienia glinianej tabliczki
 \item[Kliny] - znaki występujące na glinianych tabliczkach.
 \item[Odczyty] - sposób transkrypcji klinów, występuje na tabliczkach elektronicznych.
 \item[Pieczęć] - część tabliczki zawierająca znak rozpoznawczy autora
\end{description}

\section{Pojęcia informatyczne}
\begin{description}
 \item[Gramatyka] - formalny opis składni i struktury języka
 \item[Semantyka] - znaczenie języka
 \end{description}
\section{Pojęcia paradygmatowe}
\begin{description}
 \item[Język dziedzinowy (Domain Specific Language, DSL)] - język programowania dostosowany do dziedziny problemu, którym się zajmuje. 
 \end{description}