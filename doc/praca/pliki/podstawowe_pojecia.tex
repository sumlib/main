\chapter{Podstawowe pojęcia}\label{r:pojecia}
\section{Pojęcia dziedzinowe}
\begin{description}
 \item[Sumerolog] - naukowiec zajmujący się odczytywaniem pisma klinowego w języku sumeryjskim. Na potrzeby tej pracy
		      to pojęcie jest rozszerzone do wszystkich ludzi zajmujących się odczytywaniem tabliczek klinowych (także w innych językach) i czerpiących z nich wiedzę historyczną.
 \item[Tabliczka] - w tej pracy tabliczka będzie oznaczała tabliczkę klinową w wersji elektronicznej 
		  (chyba, że zostanie zaznaczone inaczej). Dla rozróżnienia, kiedy będziemy mówić o ``prawdziwej'', 
		  glinianej tabliczce, będziemy używać pojęcia \textbf{gliniana tabliczka}
 \item[Prowiniencja] - pojęcie używane przez sumerologów, oznacza miejsce pochodzenia/znalezienia glinianej tabliczki
 \item[Kliny] - znaki występujące na glinianych tabliczkach.
 \item[Odczyty] - sposób transkrypcji klinów. Tabliczki elektroniczne są zapisane za pomocą odczytów. Na ogół jeden klin odpowiada jednemu odczytowi, ale może odpowiadać wielu różnym sekwencjom odczytów.
 \item[Pieczęć] - część tabliczki zawierająca znak rozpoznawczy autora
\end{description}

\section{Pojęcia informatyczne}
\begin{description}
 \item[Alfabet (zbiór \Sigma)] - zbiór symboli (np. słów kluczowych, znaków specjalnych, literałów), z których zbudowane są słowa - konstrukcje języka.
 \item[Słowo nad alfabetem \Sigma] - skończony ciąg symboli należących do zbioru $\Sigma $.
 \item[Język nad alfabetem \Sigma] - zbiór słów nad alfabetem $\Sigma $.
 \item[Gramatyka] - formalny sposób definiowania języka. Składa się z czterech elementów: zbioru symboli terminalnych, zbioru symboli nieterminalnych (rozłącznego ze zbiorem symboli terminalnych), symbolu startowego (należącego do zbioru symboli nieterminalnych) oraz zbioru reguł gramatyki. Zbiór symboli terminalnych jest to alfabet, nad którym zdefiniowany jest język. Reguły gramatyki są postaci $S1 \rightarrow S2$, gdzie $S1$ i $S2$ to ciągi symboli terminalnych i nieterminalnych, przy czym w ciągu $S1$ musi wystąpić przynajmniej jeden symbol nieterminalny. Reguły określają możliwe podstawienia symboli w wyprowadzanym słowie - ciąg $S1$ można zastąpić przez $S2$. Wyprowadzanie rozpoczynamy od symbolu startowego i kończymy, gdy wszystkie symbole w danym słowie należą do zbioru symboli terminalnych. Język określony przez gramatykę jest to zbiór słów, które są możliwe do wyprowadzenia z symbolu startowego za pomocą reguł gramatyki.
 \item[Struktura leksykalna języka] - definicja symboli terminalnych (alfabetu) oraz symboli nieterminalnych (pomocniczych).
 \item[Struktura składniowa języka] - opis składni języka, zapisany np. za pomocą reguł gramatyki.
 \item[Semantyka] - znaczenie i funkcja poszczególnych konstrukcji języka.
 \end{description}
\section{Pojęcia paradygmatyczne}
\begin{description}
 \item[Język dziedzinowy (ang. domain-specific language, DSL)] - język programowania szczególnego zastosowania, rozwiązujący specyficzny problem lub zajmujący się wąską dziedziną, stworzony specjalnie na potrzeby danej dziedziny i do niej dostosowany.
 
 %język programowania dostosowany do dziedziny problemu, którym się zajmuje. 
 \end{description}