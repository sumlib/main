\chapter{Problem przeszukiwania bazy tabliczek sumeryjskich}
\section{Podstawowe pojęcia}\label{r:pojecia}
\subsection{Pojęcia dziedzinowe}
\begin{description}
 \item[Sumerolog] -- naukowiec zajmujący się odczytywaniem pisma klinowego w języku sumeryjskim. Na potrzeby tej pracy
		      to pojęcie jest rozszerzone do wszystkich ludzi zajmujących się odczytywaniem tabliczek klinowych 
		      (także w innych językach) i czerpiących z nich wiedzę historyczną.
 \item[Tabliczka (ang. tablet)] -- w tej pracy tabliczka będzie oznaczała tabliczkę klinową w wersji elektronicznej 
		  (chyba, że zostanie zaznaczone inaczej). Dla rozróżnienia, kiedy będziemy mówić o ``prawdziwej'', 
		  glinianej tabliczce, będziemy używać pojęcia \textbf{gliniana tabliczka}.
 \item[Proweniencja (ang. provenience)] -- pojęcie używane przez sumerologów, oznacza miejsce pochodzenia/znalezienia 
		    glinianej tabliczki.
 \item[Kliny (ang. cunes)] -- znaki występujące na glinianych tabliczkach.
 \item[Odczyty (ang. readings)] -- sposób transkrypcji klinów. Tabliczki elektroniczne są zapisane za pomocą odczytów. 
Na ogół jeden klin odpowiada jednemu odczytowi, ale może odpowiadać wielu różnym sekwencjom odczytów. 
Dodatkowo jeden odczyt może być zapisany za pomocą sekwencji klinów.
 \item[Pieczęć (ang. seal)] -- część tabliczki zawierająca znak rozpoznawczy autora.
\end{description}

\subsection{Pojęcia informatyczne}
\begin{description}
 \item[Alfabet (zbiór $\Sigma$, zbiór symboli terminalnych)] -- zbiór symboli (np. słów kluczowych, znaków specjalnych, literałów), 
				  z których zbudowane są słowa -- konstrukcje języka.
 \item[Zbiór symboli nieterminalnych] -- zbiór symboli pomocnicznych, rozłączny z alfabetem.
 \item[Słowo nad alfabetem $\Sigma$] -- skończony ciąg symboli należących do zbioru $\Sigma $.
 \item[Język nad alfabetem $\Sigma$] -- zbiór słów nad alfabetem $\Sigma $.
 \item[Reguły gramatyki] -- reguły definiujące sposób tworzenia słów nad danym alfabetem. Każda reguła jest postaci 
			$S1 \rightarrow S2$ , gdzie $S1$ i $S2$ to ciągi symboli terminalnych i nieterminalnych, 
			przy czym w ciągu $S1$ musi wystąpić przynajmniej jeden symbol nieterminalny. 
			Reguły określają możliwe podstawienia symboli w wyprowadzanym słowie -- ciąg $S1$ można zastąpić przez $S2$. 
 \item[Gramatyka] -- formalny sposób definiowania języka. Składa się z czterech elementów: zbioru symboli terminalnych, 
		     zbioru symboli nieterminalnych, symbolu startowego (należącego do zbioru symboli nieterminalnych) oraz 
		     zbioru reguł gramatyki. Wyprowadzanie słowa należącego do języka rozpoczynamy od symbolu startowego, 
		     przeprowadzamy podstawienia zgodnie z regułami gramatyki i kończymy, gdy wszystkie symbole w słowie 
		     należą do zbioru symboli terminalnych. Język określony przez gramatykę jest to zbiór słów, które są możliwe 
		     do wyprowadzenia z symbolu startowego za pomocą reguł gramatyki.
 \item[Struktura leksykalna języka] -- definicja symboli terminalnych (alfabetu).
 \item[Struktura składniowa języka] -- opis składni języka, zapisany np. za pomocą reguł gramatyki.
 \item[Semantyka] -- znaczenie i funkcja poszczególnych konstrukcji języka.
 \end{description}
\subsection{Pojęcia paradygmatyczne}
\begin{description}
 \item[Język dziedzinowy (ang. domain--specific language, DSL)] -- język programowania \linebreak \mbox{szczególnego} zastosowania, 
		  rozwiązujący specyficzny problem lub zajmujący się wąską dziedziną, stworzony specjalnie na potrzeby danej 
		  dziedziny i do niej dostosowany.
 
 %język programowania dostosowany do dziedziny problemu, którym się zajmuje. 
 \end{description}