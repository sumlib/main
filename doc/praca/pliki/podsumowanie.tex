\chapter*{Podsumowanie}
\addcontentsline{toc}{part}{Podsumowanie}
% Zakończenie pracy magisterskiej jest jej podsumowaniem, w którym powinno się znaleźć potwierdzenie 
% lub też obalenie założonej przez autora tezy, poparte odpowiednimi argumentami. 
% Zamieścić tu można również pewien bilans swojej pracy oraz wszelkie nasuwające się wnioski. 
% Zakończenie ma postać klamry, spinającej całą pracę magisterską w pewną całość składającą się ze wstępu, 
% rozwinięcia oraz zakończenia, które stanowi konkluzję dla całej pracy.
% 
% W zakończeniu powinno się zamieścić kilka zdań od siebie, na przykład dotyczących prognozy analizowanego zjawiska, 
% rokować, wątpliwości itp. Tak jak i we wstępie, również tutaj autor powinien popisać się niejako swoimi umiejętnościami 
% pisarskimi, erudycją i kompetencjami w opisywanej dziedzinie, istnieje bowiem duże ryzyko, iż właśnie zakończenie zostanie 
% przeczytane przez recenzenta.

\section*{Podsumowanie projektu}
\addcontentsline{toc}{section}{Podsumowanie projektu}
%cel osiągnięty
Przedstawiona w niniejszej pracy realizacja języka TQL spełnia najważniejsze postulaty. 
Po pierwsze jest on intuicyjny i prosty w użyciu dla osób znających jedynie dziedzinę problemu.
% (co potwierdza opinia dr hab. Marka Stępnia)
Po drugie minimalnie ogranicza siłę wyrazu pozwalając na tworzenie skomplikowanych zapytań.

%wady
Język TQL ma jednak kilka ograniczeń w stosunku do typowych języków zapytań. 
Największym z nich jest brak wpływu na postać wyniku, co utrudnia np. zbieranie danych statystycznych 
(m. in. nie da się spytać o ilość tabliczek spełniających dane kryteria). 
Można jednak stworzyć narzędzia do samej prezentacji wyników zapytań, które pokonają to ograniczenie. 

% TODO: niby jest to zgodne z założeniem, ale może warto zwrócić na to uwagę?
% Poza tym TQL jest jedynie językiem wyszukiwania - nie można z jego pomocą zmieniać danych znajdujących się w bazie, 
% ani dodawać nowych. Akceptujemy to, gdyż zgodnie z definicją języków dziedzinowych, TQL jest językiem wąskiego zastosowania.

%zalety
Dzięki specyficznym cechom języka TQL, można również dostosować go do wykorzystania w innych dziedzinach. 
Należy tu wspomnieć o braku ograniczenia ilości i nazw pól, po których można wyszukiwać oraz przystosowaniu 
do wyszukiwania wg kryteriów dotyczących danych tekstowych. 
Na podstawie TQL można łatwo stworzyć rodzinę języków dla różnych dziedzin zajmujących się przeszukiwaniem tekstów.

%luźne uwagi
W ramach niniejszej pracy zaprezentowane zostały dwie implementacje języka TQL. 
Po stworzeniu pierwszej czas poświęcony na drugą był już niewielki. 
Największym zadaniem było przetłumaczenie poszczególnych konstrukcji TQL na XQuery. 
Pozwala to sądzić, że każda kolejna implementacja, dla różnych typów baz i różnych schematów danych, 
% TODO: tak lepiej?
nie będzie wymagała dużego nakładu pracy. 



\section*{Możliwości rozwoju}
\addcontentsline{toc}{section}{Możliwości rozwoju}
%co dalej
% TODO: nie jestem pewna, czy powinnyśmy pisać co chcemy, czy co można z tym zrobić, bez zaznaczania, że to będziemy my (-;
%sprytne, popieram :)
W pierwszej kolejności należy stworzyć własną stronę internetową z wyszukiwarką tabliczek, 
która byłaby dostępna dla Wydziału Historii UW i rozwijana z pomocą z naukowców z tej instytucji. 
%TODO: właściwie już jest zaczęte. Wspomnieć?
Poza tym warto spróbować nawiązać współpracę z projektem CDLI, (zob. rozdział 3.).
% TODO: napisać, że wymieniliśmy już wstępne maile? Nie wykazali się dużym zainteresowaniem, 
ale może nie przedstawiłyśmy jasno o co chodzi?
Implementacja TQL dla bazy CDLI, wraz z interfejsem www, z pewnością byłaby przydatnym narzędziem dla sumerologów na całym świecie.

W dalszej perspektywie można rozwinąć język TQL między innymi o wyszukiwanie wg klinów i wg tagów.
% TODO: napisać, że wyszukiwanie wg klinów wymaga tłumaczenia odczyty -> kliny a wg. tagów narzędzia do ich wykrywania lub 
% zaznaczania przez sumerologów
Dodatkowo warto stworzyć narzędzie do analizy uszkodzonych fragmentów i wykrywania błędów w odczytach tekstów. 
Dzięki niemu można by zmniejszyć ryzyko pominięcia istotnych tabliczek podczas wyszukiwania. 

Rozwój projektu będzie zależał również od potrzeb zgłaszanych przez samych sumerologów.


% Jeśli TQL spotka się z zainteresowaniem naukowców, będziemy rozwijać język dodając przede wszystkim wyszukiwanie po klinach i po tagach. Dodatkowo chciałybyśmy stworzyć narzędzie do analizy uszkodzonych fragmentów i wykrywania pomyłek (np. literówek) w odczytanych tekstach. Bez takiego narzędzia można przy wyszukiwaniu pominąć cenne tabliczki zawierające pożądaną treść. Dalszy rozwój projektu będzie zależał przede wszystkim od potrzeb zgłaszanych przez sumerologów.



%rozszerzenie języka o nowe konstrukcje jest trudne, 


%Wnioski:
%=> Drugą implementację pisało się znacznie łatwiej
%=> udało się osiągnąć założony cel pracy
%=> istnieje potencjalna możliwość innego wykorzystania programu
%=> nie ograniczamy ilości pól, po których można wyszukiwać
%=> cechy TQL:
%      prosty i intuicyjny przez osoby znające jedynie dziedzinę problemu, -> spytać stępnia 
%      przystosowany głównie do tekstów, -> tak
%      minimalnie zmniejsza siłę wyrazu -> nie pozwala na insert i update, ale takie były założenia, poza tym tak
%      łatwo go rozbudowywać. -> rozszerzenie języka o nowe konstrukcje jest trudne, a o nowe pola proste
%      użytkownik nie ma wpływu na postać wyniku - język pozwala tylko na określenie kryteriów wyszukiwania,
% w związku z czym raczej nie nadaje się do statystyk -
%
%
%Jesteśmy z siebie dumne
%
%co dalej z tym projektem? cdli, własna wyszukiwarka?