\chapter*{Podsumowanie}
\addcontentsline{toc}{chapter}{Podsumowanie}
% Zakończenie pracy magisterskiej jest jej podsumowaniem, w którym powinno się znaleźć potwierdzenie 
% lub też obalenie założonej przez autora tezy, poparte odpowiednimi argumentami. 
% Zamieścić tu można również pewien bilans swojej pracy oraz wszelkie nasuwające się wnioski. 
% Zakończenie ma postać klamry, spinającej całą pracę magisterską w pewną całość składającą się ze wstępu, 
% rozwinięcia oraz zakończenia, które stanowi konkluzję dla całej pracy.
% 
% W zakończeniu powinno się zamieścić kilka zdań od siebie, na przykład dotyczących prognozy analizowanego zjawiska, 
% rokować, wątpliwości itp. Tak jak i we wstępie, również tutaj autor powinien popisać się niejako swoimi umiejętnościami 
% pisarskimi, erudycją i kompetencjami w opisywanej dziedzinie, istnieje bowiem duże ryzyko, iż właśnie zakończenie zostanie 
% przeczytane przez recenzenta.

Wnioski:
=> Drugą implementację pisało się znacznie łatwiej
=> udało się osiągnąć założony cel pracy
=> istnieje potencjalna możliwość innego wykorzystania programu
=> nie ograniczamy ilości pól, po których można wyszukiwać
=> cechy TQL:
      prosty i intuicyjny przez osoby znające jedynie dziedzinę problemu, -> spytać stępnia 
      przystosowany głównie do tekstów, -> tak
      minimalnie zmniejsza siłę wyrazu -> nie pozwala na insert i update, ale takie były założenia, poza tym tak
      łatwo go rozbudowywać. -> rozszerzenie języka o nowe konstrukcje jest trudne, a o nowe pola proste
      użytkownik nie ma wpływu na postać wyniku - język pozwala tylko na określenie kryteriów wyszukiwania,
 w związku z czym raczej nie nadaje się do statystyk -


Jesteśmy z siebie dumne

co dalej z tym projektem? cdli, własna wyszukiwarka?