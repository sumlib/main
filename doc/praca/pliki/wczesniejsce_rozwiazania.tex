\chapter{Wcześniejsze rozwiązania}\label{r:losers}
W chwili obecnej nie ma czegoś takiego jak język dostosowany do potrzeb sumerologów. Są strony internetowe oferujące wyszukiwanie, 
jak np.
\begin{itemize}
\item \textbf{The Cuneiform Digital Library Initiative} (http://cdli.ucla.edu) - największa znana nam baza tekstów sumeryjskich, 
wyszukiwanie po praktycznie wszystkich możliwych parametrach, choć trochę mało wygodne. Brakuje wyjaśnienia jak 
używać ``Advanced search syntax''
\item \textbf{The Electronic Text Corpus of Sumerian Literature} (http://etcsl.orinst.ox.ac.uk/) - baza znacznie mniejsza, zawiera 
głównie teksty literackie. Wyszukiwanie mało rozbudowane.
\end{itemize}
