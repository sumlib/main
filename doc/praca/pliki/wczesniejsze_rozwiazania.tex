\chapter{Wcześniejsze rozwiązania}\label{r:losers}
W chwili obecnej nie ma czegoś takiego jak język dostosowany do potrzeb sumerologów. 
Są strony internetowe oferujące wyszukiwanie za pomocą formularzy.
\section{The Cuneiform Digital Library Initiative \cite{cdli}}
\begin{figure}[h]
 \centering
 \includegraphics[width=300px]{../diagramy/cdli-search.png}
 % cdli-search.png: 877x501 pixel, 72dpi, 30.94x17.67 cm, bb=0 0 877 501
 \caption{Formularz wyszukiwania na CDLI}
 \label{fig:cdli-search}
\end{figure}

Największa znana nam baza tekstów sumeryjskich (ok. 225 tys. tekstów), 
wyszukiwanie po praktycznie wszystkich możliwych parametrach, choć trochę mało wygodne. 
Dla każdej metadanej jest pole tekstowe z możliwymi opcjami wyszukiwania: ``begins with``, ``contains'', ''does not contain''. 
Dla treści jest pole tekstowe z opcjami ``word``, ``part of word`` oraz checkbox ''advanced search syntax''. 
Brakuje wyjaśnienia jak 
używać ``Advanced search syntax'' oraz możliwości tworzenia złożonych zapytań.

\section{The Electronic Text Corpus of Sumerian Literature \cite{etcsl}} 
\begin{figure}[h]
 \centering
 \includegraphics[width=300px]{../diagramy/etcsl-search.png}
 % etcsl-search.png: 774x182 pixel, 72dpi, 27.31x6.42 cm, bb=0 0 774 182
 \caption{Formularz wyszukiwania na etcsl}
 \label{fig:etcsl-search}
\end{figure}

Baza znacznie mniejsza, zawiera 
głównie teksty literackie. Wyszukiwanie mało rozbudowane. Można wybrać:
\begin{itemize}
 \item typ wyszukiwanego słowa (form, lemma, label, pos, emesal, sign)
 \item sortowanie
 \item kategorię (ponad 20 kategorii)
 \item sposób wyświetlania wyników
\end{itemize}

Można zadać zapytanie z kilkoma słowami różnych typów, można określić czy słowo ma być rzeczownikiem, imieniem boga itp. Można też pytać o część słowa.