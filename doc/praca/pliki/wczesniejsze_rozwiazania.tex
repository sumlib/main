\chapter{Wcześniejsze rozwiązania}\label{r:losers}
W chwili obecnej nie istnieje język dostosowany do potrzeb sumerologów. Jedyne znane nam rozwiązania przedstawionego wyżej problemu opierają się o pomysł formularzy, które w łatwy sposób można wypełniać i łatwo na ich podstawie tworzyć zapytania. Wadą takich rozwiązań jest brak możliwości tworzenia skomplikowanych zapytań.
Poniżej przedstawimy dwie przykładowe strony internetowe oferujące wyszukiwanie za pomocą formularzy.
\section{The Cuneiform Digital Library Initiative \cite{cdli}}
\begin{figure}[h]
 \centering
 \includegraphics[width=300px]{../diagramy/cdli-search.png}
 % cdli-search.png: 877x501 pixel, 72dpi, 30.94x17.67 cm, bb=0 0 877 501
 \caption{Formularz wyszukiwania na stronie CDLI}
 \label{fig:cdli-search}
\end{figure}

CDLI to największa znana nam baza tekstów sumeryjskich, zawiera ok. 225 tys. tekstów. 
Umożliwia wyszukiwanie po wielu parametrach.
Dla każdej metadanej jest pole tekstowe z możliwymi opcjami wyszukiwania: ``begins with``, ``contains'', ''does not contain''. 
Dla treści jest pole tekstowe z opcjami ``word``, ``part of word``.
Jest również checkbox ''advanced search syntax'', jednak brakuje wyjaśnienia jak go używać.
Nie można tworzyć bardziej skomplikowanych warunków dla poszczególnych parametrów oraz zapytań złożonych.

\section{The Electronic Text Corpus of Sumerian Literature \cite{etcsl}} 
\begin{figure}[h]
 \centering
 \includegraphics[width=300px]{../diagramy/etcsl-search.png}
 % etcsl-search.png: 774x182 pixel, 72dpi, 27.31x6.42 cm, bb=0 0 774 182
 \caption{Formularz wyszukiwania na stronie ETCSL}
 \label{fig:etcsl-search}
\end{figure}

ETCSL jest znacznie mniejszą bazą, zawierającą głównie teksty literackie. 
Ma ograniczone możliwości wyszukiwania po metadanych (tylko po kategorii tekstu), jednak udostępnia tworzenie bardziej 
skomplikowanych zapytań dotyczące treści tabliczki. 
Pozwala określić typ wyszukiwanego słowo (form, lemma, label, pos, emesal, sign),
 a także jego znaczenie w tekście (np. czy jest imieniem bóstwa) oraz część mowy, do której należy. 
Można również wyszukiwać po fragmencie słowa.
